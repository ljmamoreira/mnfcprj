\section{Introduction and first steps}
This document is the stage for a small \LaTeX{} collaborative practice
experiment with \texttt{git}.

\subsection{Creating and using a local repository}
To start it, I created a new directory (aka folder) named \texttt{mnfcprj} and
issued the command \texttt{git init} at a terminal opened at that dir. This
command created my local version of the \gls{repo}.

Then, I created the \LaTeX{} main file (\texttt{/mnfcprj.tex}), a different
directory to populate with different chapter files to be
\texttt{\textbackslash{}include}'d and started with the first chapter file
(\texttt{intro.tex}).

Also, I thought that it might be usefull to add a glossary to the document; for
that I created a file to store the glossary definitions (\texttt{glossary.def})
at the root dir. 

Finally, I created the \texttt{.gitignore} file (the initial dot makes it hidden
in unix systems), listing the files generated by the \LaTeX{} (\texttt{.aux},
etc) that we don't to keep under revision control.

The command \texttt{git status} generates the following report:
\begin{term}
\begin{verbatim}
ljma:mnfcprj$ git status
On branch master

No commits yet

Untracked files:
  (use "git add <file>..." to include in what will be committed)
	.gitignore
	chpts/
	glossary.def
	mnfcprj.tex

nothing added to commit but untracked files present (use "git add" to track)
\end{verbatim}
\end{term}
The output indicates that there are (or, mor acurately, there \emph{were})
several new files (not yet included in the repository) and one new dir. We tell
\texttt{git} to add these files and dirs to the directory with the command

\texttt{git add <file name> <more filenames if needed>}

This tells \git what filenames it must keep track of but they aren't yet part of
the repository. Changes to the repository content only occur after one issues
the command

\texttt{git commit}

This command starts a text editor we use to write a short description of the
commit.

\subsection{Standard local workflow}
My \emph{usual} \git workflow on an individual project under development is the
following\nopagebreak
\begin{enumerate}
  \item Do some editing to the project (add a new file, correct bugs, add some
    section to an existing file
  \item \texttt{git add} the added or modified files
  \item \texttt{git commit} and restart
\end{enumerate}

\git{} keeps track of the all the files in the project and you can recover any
previous version of any file in the repository.

\texttt{git log} lists the commit message for all commits. That can be useful to
help locating some particular previous version of some file.

\subsection{Branches}
Suppose we want to make a large change to some project. For instance, in a
\LaTeX project, we may want to merge two chapters. To that one may have to make
a lot of changes to the files and it can be very difficult to make \emph{all}
these changes. That may kleave the project in a broken state, and it may be hard
to fix it fast enough. \git{} helps us with that using the concept of
\emph{branches}. A branch can be thought as a version of the document,
independent of other branches. 

So, how merge two chapters (or do sume complex modification to your repository?
\begin{enumerate}
  \item \texttt{commit} the current version, so that you can go back to it in
    case something fails
  \item Create a new branch and set it current
  \item Edit, add, commit (to the new branch), until all the changes are done
  \item Test, test test, until you are confidente that the new version is
    satisfactory
  \item \emph{Merge} the two branches
\end{enumerate}

\section{Homework}
Browse the online \git{} manuals and tutorials to learn the commands (and their
syntax) that perform these basic tasks (\texttt{git add}, \texttt{git commit},
\texttt{git branch}, \texttt{git merge} and more). Use this knowledge to add
some section to file \texttt{chpts/intro.tex}. But first you must \emph{clone}
the project to your computer. I hosted the repository for this project at
github. To clone it tou your computer so that you can work on it, go to an
apropriate folder of your choosing and issue the command

\texttt{git clone https://github.com/ljmamoreira/mnfcprj}

This command will create a copy of the project on your computer. (Later, well
talk again about \texttt{git clone}.) Then you can
experiment with \git. Please do, add another section to the \texttt{intro.tex}
file, add and commit. Please let me know if you have any difficulties.
